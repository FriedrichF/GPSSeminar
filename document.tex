%%%%%%%%%%%%%%%%%%%%%%%%%%%%%%%%%%%%%%%%%%%%%%%%%%%%%%%%%%%%%%%%%%%%%%%%%%%%%%
%                  Latex Vorlage f\"{u}r Abschlu{\ss}arbeiten                        %
%%%%%%%%%%%%%%%%%%%%%%%%%%%%%%%%%%%%%%%%%%%%%%%%%%%%%%%%%%%%%%%%%%%%%%%%%%%%%%
\documentclass[12pt,a4paper,dvips,DIV13,BCOR10mm,fleqn,liststotoc,bibtotoc,cleardoubleempty]{scrbook}
%,footexclude,headexclude standard,chapterprefix,
\usepackage{times}            %Font
\usepackage[latin1]{inputenc} %erkennt \"{a},\"{o},\"{u}
\usepackage{german,amsfonts}  %erkennt \3 als {\ss}
\usepackage[T1]{fontenc}      %z.B. f\"{u}r (besseres) automatisches Trennen nach Umlauten
\usepackage{latexsym}         %sonstige Symbole (\partial)
\usepackage{graphicx}         %[draft]
%\usepackage[a4,center]{crop}  % cam,frame, habe D:\texmf\dvips\config\config.ps angepa{\ss}t!!!
\usepackage{pstricks,pst-node,pst-tree,pst-plot}  %um B\"{a}ume zu zeichnen
\usepackage{booktabs}         %f\"{u}r sch\"{o}nere Tabellen
\usepackage{listings}         %fuer C-Programme
\usepackage{textcomp}         %f\"{u}r \textperthousand = promille
\usepackage{placeins}         %verhindert Gleiten von Abbildungen (bis \FloatBarrier)
\usepackage{scrpage2}%[automark] f\"{u}r Kopfzeilen
\setheadsepline{0.4pt} %
\pagestyle{scrheadings} %Linie in Kopfzeilen
\addtolength{\topmargin}{5mm} %gesamten Text auf der Seite nach unten schieben
%\usepackage{setspace} Zeilenabstand \"{a}ndern. H\"{a}nde weg! Vergr\"{o}{\ss}ert auch den Abstand von Bildunterschriften...
%\onehalfspacing
%%%%%%%%%%%%%%%%%%%%%%%%%%%%%%%%%%%%%%%%%%%%%%%%%%%%%%%%%%%%%%%%%%%%%%%%%%%%%%
%%%%%%%%%%%%%%%%%%%%%%%%%%%%%%%%%%%%%%%%%%%%%%%%%%%%%%%%%%%%%%%%%%%%%%%%%%%%%%
\begin{document}
%%%%%%%%%%%%%%%%%%%%%%%%%%%%%%%%%%%%%%%%%%%%%%%%%%%%%%%%%%%%%%%%%%%%%%%%%%%%%%
%%%%%%%%%%%%%%%%%%%%%%%%%%%%%%%%%%%%%%%%%%%%%%%%%%%%%%%%%%%%%%%%%%%%%%%%%%%%%%
\frontmatter
%von http://studiy.tu-cottbus.de/projektwiki/wissen:latex:diplomarbeit
\titlehead{%  {\centering Seitenkopf}
  {Hochschule f\"{u}r angewandte Wissenschaften\\
   Fachhochschule W\"{u}rzburg-Schweinfurt\\
   Fakult\"{a}t Informatik und Wirtschaftsinformatik}}
\subject{Bachelorseminar}
\title{GPS - Energieverbrauch in Smartphones}
\subtitle{\normalsize{vorgelegt an der Hochschule f\"{u}r angewandte
Wissenschaften Fachhochschule W\"{u}rzburg-Schweinfurt in der Fakult\"{a}t
Informatik und Wirtschaftsinformatik f�r das Schwerpunktseminar im Bereich der
Technischen Informatik}}
\author{Sascha Greiner-Adam und Friedrich Fell}
\date{\normalsize{\today}}
\publishers{
  \normalsize{Erstpr\"{u}fer: Prof. Dr. Balzer} \\
  \normalsize{Zweitpr\"{u}fer: Prof. Dr. Braun}\\
}

%\uppertitleback{ }
%\lowertitleback{ }

\maketitle

\thispagestyle{empty}
\section*{Selbstst\"{a}ndigkeitserkl\"{a}rung}
Hiermit versichere ich, dass ich die vorgelegte Arbeit selbstst\"{a}ndig
verfasst und noch nicht anderweitig zu Pr\"{u}fungszwecken vorgelegt habe. Alle
benutzten Quellen und Hilfsmittel sind angegeben, w\"{o}rtliche und
sinngem\"{a}{\ss}e Zitate wurden als solche gekennzeichnet.\\[15mm]
%% Abstand und Linie
\vspace{20mm}
\hrule
\vspace{5mm}
W�rzburg, den \date{\today}

\vspace{5cm}

\thispagestyle{empty}
\section*{Kurzfassung}
In dieser Arbeit geht es darum den Energieverbrauch von Smartphones bei GPS
benutzung zu analysieren und Energiesparendere alternativen zu finden.

\vspace{5cm}

\nonfrenchspacing
\renewcommand{\figurename}{Abb.}
\renewcommand{\tablename}{Tab.}

%\setcounter{page}{7}%  {tocdepth}{1} =section
%\lstset{language=C, basicstyle=\ttfamily\small, commentstyle=\itshape}
%\automark[chapter]{chapter} ???

\tableofcontents %
\listoffigures %
\listoftables
%%%%%%%%%%%%%%%%%%%%%%%%%%%%%%%%%%%%%%%%%%%%%%%%%%%%%%%%%%%%%%%%%%%%%%%%%%%%%%
%%%%%%%%%%%%%%%%%%%%%%%%%%%%%%%%%%%%%%%%%%%%%%%%%%%%%%%%%%%%%%%%%%%%%%%%%%%%%%
%%%%%%%%%%%%%%%%%%%%%%%%%%%%%%%%%%%%%%%%%%%%%%%%%%%%%%%%%%%%%%%%%%%%%%%%%%%%%%
%%%%%%%%%%%%%%%%%%%%%%%%%%%%%%%%%%%%%%%%%%%%%%%%%%%%%%%%%%%%%%%%%%%%%%%%%%%%%%
\mainmatter

\chapter{Einleitung}
Das kommt hier

\chapter{Theoretische Ans\"{a}tze zur Reduzierung des Energieverbrauch}
Hier werden verschiedene verfahren aufgezeigt wie der Energieverbrauch in
Smartphones durch Ortsbestimmung minimiert werden kann.
\section[Ephemeriden/Almanach Download]{Download der Ephemeriden/Almanach
Daten \"{u}ber alternative Verbindungen}

\section[Abfrageh�ufigkeit durch GPS Position]{Abfrageh\"{a}ufigkeit anhand der
Ver\"{a}nderung der GPS Position festlegen}

\section[Abfrageh�ufigkeit durch Bewegungssensoren]{Abfrageh\"{a}ufigkeit anhand
der Bewegungssensoren feststellen}

\section[Sensordaten verbessern]{Verbesserung der Sensordaten durch Kalman
Filter}

\section[Alternative Positionsbestimmung]{Positionsbestimmung durch Sensordaten
und seltener Abgleich durch GPS}

\chapter{Praktische Ermittlung des Energieverbrauchs bei der GPS Nutzung}

\section[Toolauswahl]{Toolauswahl zum Energieverbrauch}

\section[Testumgebung auf Smartphone definieren]{Testumgebung auf Smartphone
definieren}

\section[Kommunikation eingehen]{Kommunikation eingehen}

\section[Verschiedene Szenarien bei GPS zur
Energieverbrauchsbestimmung]{Verschiedene Szenarien bei GPS zur Energieverbrauchsbestimmung}

\section[Energieverbrauch bei GPS-Alternativen vergleichen]{Energieverbrauch bei
GPS-Alternativen vergleichen}


Hier wird eine Buch zitiert \cite{book-minimal} und hier ein anderes \cite{article-full}.

%In der Realit\"{a}t sollten die Kapitel in einzelnen Dateien gespeichert werden
% und diese mit \include eingebunden werden.
%\include{einleitung}
%
%\include{zusammenfassung}

%%%%%%%%%%%%%%%%%%%%%%%%%%%%%%%%%%%%%%%%%%%%%%%%%%%%%%%%%%%%%%%%%%%%%%%%%%%%%%
%%%%%%%%%%%%%%%%%%%%%%%%%%%%%%%%%%%%%%%%%%%%%%%%%%%%%%%%%%%%%%%%%%%%%%%%%%%%%%
%%%%%%%%%%%%%%%%%%%%%%%%%%%%%%%%%%%%%%%%%%%%%%%%%%%%%%%%%%%%%%%%%%%%%%%%%%%%%%
%%%%%%%%%%%%%%%%%%%%%%%%%%%%%%%%%%%%%%%%%%%%%%%%%%%%%%%%%%%%%%%%%%%%%%%%%%%%%%
\backmatter
%\addpart{Anhang}
%\appendix
%\include{anhang_hard}
%\addcontentsline{toc}{\bibliography}

\bibliographystyle{galpha2a} %geralpha, dinat, dalpha, ...
\bibliography{gerxampl}

\end{document}

